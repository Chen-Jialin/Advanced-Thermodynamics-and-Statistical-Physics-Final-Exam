\documentclass{assignment}
\ProjectInfos{高等热力学与统计物理}{PHYS2110}{2020-2021学年第二学期}{期末考试}{截止时间:2021. 6. 16(周三)12:00}{陈稼霖}{45875852}

\begin{document}
\begin{itemize}
    \item 通用符号:$T=$ 温度,$P=$ 压强,$V=$ 体积,$N=$ 粒子数,$\rho=$ 数密度,$\mu=$ 化学势,$k=$ Boltzmann 常数,$\hbar=$ Plank 常数.
    \item 热力学极限:
    \[
        N\rightarrow\infty\quad V\rightarrow\infty\quad\rho=\frac{N}{V}\neq 0\text{且有限}.
    \]
\end{itemize}
\begin{prob}
    某一磁性物质对外界所做的微功为
    \[
        \delta W=-H\mathrm{d}M
    \]
    其中 $H$ 和 $M$ 为磁场强度和磁化强度. 体积固定且设为 $1$.\\
    如果 $H$, $M$ 和 温度 $T$ 的关系为
    \[
        M=\frac{aH}{T-T_c}
    \]
    $a$, $T_c$ 为常数且 $T>T_c$.
    \begin{itemize}
        \item[1)] 证明该物质的内能由下式给出:(10 分)
        \[
            U(T,M)=U(T,0)-\frac{M^2T_c}{2a}.
        \]
        \item[2)] 求该物质在 $H$ 固定条件下的热容量,$C_H(T,M)$.(10 分)
    \end{itemize}
\end{prob}
\begin{sol}
    \begin{itemize}
        \item[1)] 由热力学第一定律的微分形式
        \begin{align}
            \mathrm{d}U=\delta Q-\delta W=T\,\mathrm{d}S+H\,\mathrm{d}M.
        \end{align}
        以及 Helmholtz 自由能的定义
        \begin{align}
            F=U-TS,
        \end{align}
        可得 Helmholtz 自由能的微分为
        \begin{align}
            \mathrm{d}F=-S\,\mathrm{d}T+H\,\mathrm{d}M.
        \end{align}
        由上式可得
        \begin{align}
            S=&-\left(\frac{\partial F}{\partial T}\right)_M,\\
            H=&\left(\frac{\partial F}{\partial M}\right)_T.
        \end{align}
        以上两式分别关于磁化强度 $M$ 和 温度 $T$ 求导可得
        \begin{align}
            \label{1-dS/dM=-dH/dT}
            \left(\frac{\partial S}{\partial M}\right)_T=-\frac{\partial^2F}{\partial M\partial T}=-\left(\frac{\partial H}{\partial T}\right)_M.
        \end{align}

        在给定温度下,磁性物质的内能关于磁化强度 $M$ 的偏导为
        \begin{align}
            \left(\frac{\partial U}{\partial M}\right)_T=T\left(\frac{\partial S}{\partial M}\right)_T+H.
        \end{align}
        将式 \eqref{1-dS/dM=-dH/dT} 以及 $H$, $M$ 和 $T$ 的关系 $M=\frac{aH}{T-T_c}$,代入上式可得
        \begin{align}
            \label{1-dU/dM}
            \left(\frac{\partial U}{\partial M}\right)_T=-T\left(\frac{\partial H}{\partial T}\right)_M+H=-\frac{TM}{a}+\frac{(T-T_c)M}{a}=-\frac{T_cM}{a}.
        \end{align}
        因此,
        \begin{align}
            U(T,M)=U(T,0)+\int_0^M\left(\frac{\partial U}{\partial M}\right)_T\,\mathrm{d}M=U(T,0)-\int_0^M\frac{T_cM'}{a}\,\mathrm{d}M'=U(T,0)-\frac{T_cM^2}{2a}.
        \end{align}
        \item[2)] 该物质在 $H$ 固定条件下的热容量为
        \begin{align}
            C_H(T,M)=&\lim_{\Delta T\rightarrow 0}\left(\frac{\Delta Q}{\Delta T}\right)_H=\lim_{\Delta T\rightarrow 0}\left(\frac{\Delta U-H\,\mathrm{d}M}{\Delta T}\right)_H=\frac{\mathrm{d}U(T,M=0)}{\mathrm{d}T}-\frac{T_cM}{a}\left(\frac{\partial M}{\partial T}\right)_H-H\left(\frac{\partial M}{\partial T}\right)_H\\
            =&\frac{\mathrm{d}U(T,M=0)}{\mathrm{d}T}+\left(\frac{T_cM}{a}+H\right)\frac{aH}{(T-T_c)^2}=\frac{\mathrm{d}U(T,M=0)}{\mathrm{d}T}+\frac{TM^2}{a(T-T_c)}.
        \end{align}
        % 一方面,由热力学第一定律以及该物质在 $H$ 固定下的热容量的定义 $C_M(T,M)=\left(\frac{\partial U}{\partial T}\right)_M$,
        % \begin{align}
        %     T\,\mathrm{d}S=\mathrm{d}U-H\,\mathrm{d}M=\left(\frac{\partial U}{\partial T}\right)_M\,\mathrm{d}T+\left(\frac{\partial U}{\partial M}\right)_T\,\mathrm{d}M-H\,\mathrm{d}M=C_M\,\mathrm{d}T+\left(\frac{\partial U}{\partial M}\right)_T\,\mathrm{d}M-H\,\mathrm{d}M.
        % \end{align}
        % 将式 \eqref{1-dU/dM} 代入上式可得
        % \begin{align}
        %     \label{1-TdS-1}
        %     T\,\mathrm{d}S=C_M\,\mathrm{d}T-T\left(\frac{\partial H}{\partial T}\right)_M\,\mathrm{d}M.
        % \end{align}
        % 另一方面,该物质的 Gibbs 势为
        % \begin{align}
        %     G=U-TS-HM.
        % \end{align}
        % 从而 Gibbs 势的微分为
        % \begin{align}
        %     \mathrm{d}G=-S\,\mathrm{d}T-M\,\mathrm{d}H.
        % \end{align}
        % 由上式得
        % \begin{align}
        %     S=&-\left(\frac{\partial G}{\partial T}\right)_H,\\
        %     M=&-\left(\frac{\partial G}{\partial H}\right)_T.
        % \end{align}
        % 以上两式分别关于磁场强度 $H$ 和熵 $S$ 求偏导可得
        % \begin{align}
        %     \label{1-dT/dH=-dM/dS}
        %     \left(\frac{\partial S}{\partial H}\right)_T=\frac{\partial^2\mathcal{H}}{\partial S\partial H}=-\left(\frac{\partial M}{\partial T}\right)_H.
        % \end{align}
        % 磁性物质的焓定义为
        % \begin{align}
        %     \mathcal{H}=U-HM,
        % \end{align}
        % 焓的微分为
        % \begin{align}
        %     \label{1-dH}
        %     \mathrm{d}\mathcal{H}=T\,\mathrm{d}S-M\,\mathrm{d}H,
        % \end{align}
        % 焓 $\mathcal{H}$ 关于磁场强度 $H$ 的偏导为
        % \begin{align}
        %     \left(\frac{\partial\mathcal{H}}{\partial H}\right)_T=T\left(\frac{\partial S}{\partial H}\right)_T-M.
        % \end{align}
        % 将式 \eqref{1-dT/dH=-dM/dS} 代入上式可得
        % \begin{align}
        %     \label{1-dH/dH}
        %     \left(\frac{\partial\mathcal{H}}{\partial H}\right)_T=-T\left(\frac{\partial M}{\partial T}\right)_H-M.
        % \end{align}
        % 该物质在 $H$ 固定条件下的热容量 $C_H(T,M)=\left(\frac{\partial U}{\partial T}\right)_H-H\left(\frac{\partial M}{\partial T}\right)_H=\left(\frac{\partial\mathcal{H}}{\partial T}\right)_H$,由式 \eqref{1-dH}
        % \begin{align}
        %     T\,\mathrm{d}S=\mathrm{d}\mathcal{H}+M\,\mathrm{d}H=\left(\frac{\partial\mathcal{H}}{\partial T}\right)_H\,\mathrm{d}T+\left(\frac{\partial\mathcal{H}}{\partial H}\right)_T\,\mathrm{d}H+M\,\mathrm{d}H=C_H\,\mathrm{d}T+\left(\frac{\partial\mathcal{H}}{\partial H}\right)_T\,\mathrm{d}H+M\,\mathrm{d}H.
        % \end{align}
        % 将式 \eqref{1-dH/dH} 以及 $H$, $M$ 和 $T$ 的关系 $M=\frac{aH}{T-T_c}$,代入上式可得
        % \begin{align}
        %     \label{1-TdS-2}
        %     T\,\mathrm{d}S=C_H\,\mathrm{d}T-T\left(\frac{\partial M}{\partial T}\right)_H\,\mathrm{d}H.
        % \end{align}

        % 式 \eqref{1-TdS-1} 和 \eqref{1-TdS-2} 联立消去 $T\,\mathrm{d}S$ 得
        % \begin{gather}
        %     (C_H-C_M)\,\mathrm{d}T=T\left[\left(\frac{\partial M}{\partial T}\right)_H\,\mathrm{d}H+\left(\frac{\partial H}{\partial T}\right)_M\,\mathrm{d}M\right],\\
        %     \label{1-dT-1}
        %     \Longrightarrow\mathrm{d}T=\frac{T}{C_H-C_M}\left[\left(\frac{\partial H}{\partial T}\right)_M\,\mathrm{d}M+\left(\frac{\partial M}{\partial T}\right)_H\,\mathrm{d}H\right]
        % \end{gather}
        % 由状态方程得
        % \begin{align}
        %     \label{1-dT-2}
        %     \mathrm{d}T=\left(\frac{\partial T}{\partial M}\right)_H\,\mathrm{d}M+\left(\frac{\partial T}{\partial H}\right)_M\mathrm{d}H
        % \end{align}
        % 比较上面两式(\eqref{1-dT-1} 和 \eqref{1-dT-2})可得
        % \begin{align}
        %     \frac{T}{C_H-C_M}\left(\frac{\partial H}{\partial T}\right)_M=&\left(\frac{\partial T}{\partial M}\right)_H,\\
        %     \frac{T}{C_H-C_M}\left(\frac{\partial M}{\partial T}\right)_H=&\left(\frac{\partial T}{\partial H}\right)_M,
        % \end{align}
        % \begin{align}
        %     \label{1-CH-CM}
        %     \Longrightarrow C_H-C_M=\frac{T\left(\frac{\partial H}{\partial T}\right)_M}{\left(\frac{\partial T}{\partial M}\right)_H}.
        % \end{align}

        % 接下来考虑绝热过程,令式 \eqref{1-TdS-1} 和 \eqref{1-TdS-2} 中的 $\mathrm{d}S=0$,有
        % \begin{align}
        %     C_M=&T\left(\frac{\partial H}{\partial T}\right)_M\left(\frac{\partial M}{\partial T}\right)_S,\\
        %     C_H=&T\left(\frac{\partial M}{\partial T}\right)_H\left(\frac{\partial H}{\partial T}\right)_S.
        % \end{align}
        % 上面两式相除可得
        % \begin{align}
        %     \label{1-CH/CM}
        %     \frac{C_H}{C_M}=\frac{\left(\frac{\partial M}{\partial T}\right)_H\left(\frac{\partial H}{\partial T}\right)_S}{\left(\frac{\partial H}{\partial T}\right)_M\left(\frac{\partial M}{\partial T}\right)_S}=-\frac{\left(\frac{\partial M}{\partial H}\right)_T}{\left(\frac{\partial M}{\partial H}\right)_S}=
        % \end{align}
        % 联立式 \eqref{1-CH-CM} 和 \eqref{1-CH/CM} 可得
        % \begin{align}
        %     C_H=-\frac{T\left[\left(\frac{\partial M}{\partial T}\right)_H\right]^2}{\left(\frac{\partial M}{\partial H}\right)_T-\left(\frac{\partial M}{\partial H}\right)_S}
        % \end{align}
        % 其中根据 $H$, $M$ 和 $T$ 的关系 $M=\frac{aH}{T-T_c}$,
        % \begin{align}
        %     \left(\frac{\partial M}{\partial T}\right)_H=&-\frac{aH}{(T-T_c)^2},\\
        %     \left(\frac{\partial M}{\partial H}\right)_T=&\frac{a}{T-T_c}.
        % \end{align}
        % 在绝热磁化中,
    \end{itemize}
\end{sol}
\clearpage

\begin{prob}
    一温度为 $T$ 的圆柱形容器被一活塞隔成两部分. 每部分放置一种非相对论 Fermi 气体. 活塞可以自由移动. 两种 Fermi 气体的分子质量相同,但自旋不同,分别为 $j_1$ 和 $j_2$. 求 $T=0$ 和 $T\rightarrow\infty$ 条件下两种 Fermi 气体分子数密度的比值.(20 分)
    \begin{figure}[h]
        \centering
        \includegraphics[width=.5\columnwidth]{P2.png}
    \end{figure}
\end{prob}
\begin{sol}
    
\end{sol}
\clearpage

\begin{prob}
    考虑一低密度的经典单原子非理想气体,原子之间的相互作用势能为
    \[
        u(r)=\left\{\begin{array}{ll}
            \infty,&r\leq a\\
            -g,&a<r<b\\
            0,&r\geq b
        \end{array}\right.
    \]
    \begin{itemize}
        \item[1)] 求精确到 $\rho^2$ 的状态方程,即把 $P/kT$ 展开到 $\rho^2$.(10 分)
        \item[2)] 求该气体的化学势对理想气体化学势的领头阶修正.(5 分)
        \item[3)] 求该气体的熵和内能对理想气体熵和内能的领头阶修正.(5 分)
    \end{itemize}
\end{prob}
\begin{sol}
    
\end{sol}
\clearpage

\begin{prob}
    用 Monte Carlo 方法数值求解零场下正方格点上的二维 Ising 模型. 假设最近邻铁磁耦合,且任何近邻对的耦合能量相同.
    \begin{itemize}
        \item[1)] 写出 Hamiltonian 和计算程序的流程.(5 分)
        \item[2)] 绘出磁化强度作为温度的函数的图像.(10 分)
        \item[3)] 确定临界温度并与严格解比较.(5 分)
    \end{itemize}
    提示:要求格点至少为 $10\times 10$,并取周期性边界条件.
\end{prob}
\begin{sol}
    
\end{sol}
\clearpage

\begin{prob}
    弱耦合自旋为零玻色气体的相互作用算符为
    \[
        \Omega=\frac{1}{2V}g\sum_{\vec{p}_1+\vec{p}_2=\vec{p}_1'+\vec{p}_2'}a_{\vec{p}_1'}^{\dagger}a_{\vec{p}_2'}^{\dagger}a_{\vec{p}_2}a_{\vec{p}_1}\quad g>0
    \]
    其中 $a_{\vec{p}}$ 和 $a_{\vec{p}^{\dagger}}$ 为动量表象的湮灭产生算符,$g$ 为耦合常数. 令 $\lvert\{n_{\vec{p}}\}\rangle$ 为自由波色气体的能量本征态,其中 $n_{\vec{p}}$ 为动量 $\vec{p}$ 态的占据数.
    \begin{itemize}
        \item[1)] 证明相互作用能密度的平均值
        \[
            E[\{n_{\vec{p}}\}]\equiv\frac{1}{V}\langle\{n_{\vec{p}}\}\rvert\Omega\lvert\{n_{\vec{p}}\}\rangle=g\rho^2-\frac{g}{2V}\sum_{\vec{p}}n_{\vec{p}}^2-\frac{g\rho}{2V}
        \]
        其中 $\rho$ 玻色子的数密度.
        \item[2)] 在热力学极限下比较有 Bose-Einstein 凝聚的上述平均值 $E$ 和没有 Bose-Einstein 凝聚的上述平均值 $E'$,证明
        \[
            E<E'\tag{\text{(5 分)}}
        \]
    \end{itemize}
\end{prob}
\begin{pf}
    
\end{pf}
\end{document}